\documentclass{tufte-handout}
\usepackage{amsmath}
\usepackage[utf8]{inputenc}
\usepackage{mathpazo}
\usepackage{booktabs}
\usepackage{microtype}
\usepackage{tikz}
\usepackage{pgfplots}
\pgfplotsset{width=9cm,compat=1.13}

\title{Foursum Report}
\author{Domenico Villani, Peter Zander Havgaard \& Thomas Keralla Høpfner-Dahl}

\begin{document}
\maketitle
\thispagestyle{empty}

\section{Exhaustive search}


Our program \texttt{Simple.java} solve the Four-sum problem using four nested loops.
The index variables $i$, $j$, $k$, $l$ run respectively from 0, 1, 2 and 3 to N.
Thus, we can bound the number of array accesses by $\sim N^4\log N$.


\subsection{Experiments}
The following table summarises the empirical performance data on the input files in the \texttt{input} directory, created using Weed.java.
We ran each input size 5 times, and report the minimum, maximum, and average running time.
The tests were run on a machine with an Intel 3570k CPU running Linux. The unix \texttt{"time"} command was used to measure the elapsed time in seconds.

\bigskip\noindent
{ \small
  \begin{tabular}{cccccccc}
  \toprule
& \multicolumn{3}{c}{Simple.java} & $\quad$    \\\cmidrule{2-4} \cmidrule{6-8}
  $N$     & $\min$     & $\max$ & avg.  \\\midrule
  30 & 0.06 & 0.08 & 0.064 \\
  50 & 0.07 & 0.08 & 0.072 \\
  100 & 0.08 & 0.09 & 0.082 \\
  200 & 0.13 & 0.14 & 0.138 \\
  400 & 0.86 & 0.87 & 0.868 \\
  800 & 3.86 & 9.19 & 8.11 \\
  1600 & 107.88 & 108.32 & 108.15 \\
  3200 & 178.46 & 178.92 & 178.79 \\
  \bottomrule
  \end{tabular}
}

\newpage

\subsection{Exhaustive Search Python}
We have also implemented the same algorithm in python as \texttt{simple.py}. The python implementation was run using python2 and only values 30 through 400 were computed in the interest of time. The table below show the recorded running times from this implementation.

\bigskip\noindent
{ \small
  \begin{tabular}{cccccccc}
  \toprule
& \multicolumn{3}{c}{simple.py} & $\quad$    \\\cmidrule{2-4} \cmidrule{6-8}
  $N$     & $\min$     & $\max$ & avg.  \\\midrule
  30 &  0.05 & 0.06 & 0.052  \\
  50 &  0.08 & 0.08 & 0.08  \\
  100 &  1.23 & 1.75 & 1.472  \\
  200 &  19.46 & 22.31 & 20.588  \\
  400 &  218.41 & 294.35 & 244.86  \\
  \bottomrule
  \end{tabular}
}

\bigskip

\section{Improved Search}
Using the binary search-based idea sketeched in [SW, 1.4] for the Three-sum problem, we can improve our running time to $\sim N^3\log N$.
We implemented this algorithm in the [\texttt{Faster.java}] document. To do so we created a method based on the same algorithm of the binary search that was able to handle long data types.

\bigskip

The following table reports our maximum, minimum and average running times on the test inputs.

\bigskip\noindent
{ \small
  \begin{tabular}{cccccccc}
  \toprule
& \multicolumn{3}{c}{Faster.java} & $\quad$    \\\cmidrule{2-4} \cmidrule{6-8}
  $N$     & $\min$     & $\max$ & avg.  \\\midrule
  30 & 0.06 & 0.07 & 0.068 \\
  50 & 0.07 & 0.08 & 0.078 \\
  100 & 0.07 & 0.08 & 0.076 \\
  200 & 0.09 & 0.1 & 0.096 \\
  400 & 0.22 & 0.23 & 0.22 \\
  800 & 1.38 & 1.54 & 1.458 \\
  1600 & 10.84 & 13.35 & 11.459 \\
  3200 & 90.5 & 94.21 & 91.429 \\
  \bottomrule
  \end{tabular}
}

\newpage

\section{Quadratic Search}
We have implemented a faster algorithm called FasterQuad.java which is able to improve our running time to $\sim N^2\log N$. It is based on a inner class called Sum which represents the sum of two numbers in our array. The inner class also keeps track of the elements which are summed using a method (noCommon) to be sure that there is no repetition in the count. Then the algorithm uses two nested for loops to compare the values.

\bigskip

The following table reports our maximum, minimum and average running times on the test inputs.

\bigskip\noindent
{ \small
  \begin{tabular}{cccccccc}
  \toprule
& \multicolumn{3}{c}{FasterQuad.java} & $\quad$    \\\cmidrule{2-4} \cmidrule{6-8}
  $N$     & $\min$     & $\max$ & avg.  \\\midrule
  30 & 0.06 & 0.08 & 0.07 \\
  50 & 0.06 & 0.08 & 0.068 \\
  100 & 0.07 & 0.08 & 0.072 \\
  200 & 0.09 & 0.1 & 0.096 \\
  400 & 0.13 & 0.13 & 0.13 \\
  800 & 0.28 & 0.32 & 0.298 \\
  1600 & 0.64 & 0.68 & 0.666 \\
  3200 & 2.95 & 3.02 & 2.983 \\
  \bottomrule
  \end{tabular}
}

\newpage

\section{Graphs}
Below are two graphs comapring the running times of the three implemented algorithms. The first is a strandard plat and second a log plot.

\bigskip\noindent

\begin{tikzpicture}
	\begin{axis}[
		title={Average running time of 5 runs},
		xlabel={$N$},
    % xmode = log,
    % log ticks with fixed point,
    % ymode = log,
		ylabel={Time (s)},
		xmin=50, xmax=800,
		ymin=0.01, ymax=1.5,
		xtick={50, 100, 200, 400, 800},
		ytick={0.01,0.5,1,2},
    legend style={at={(1.1,0)}, anchor=south west},
		ymajorgrids=true,
	]
  \addplot[ color=blue, mark=x,error bars/.cd,y dir=both,y explicit ] table [x index=0, y index=1] {output/tex_tables/Simple.table};
  \addplot[ color=red,  mark=*,error bars/.cd,y dir=both,y explicit ]	table [x index=0, y index=1] {output/tex_tables/Faster.table};
  \addplot[ color=green,mark=o,error bars/.cd,y dir=both,y explicit ] table [x index=0, y index=1] {output/tex_tables/FasterQuad.table};
  \addplot[ color=orange,mark=o,error bars/.cd,y dir=both,y explicit ] table [x index=0, y index=1] {output/tex_tables/Simple_python.table};
	\legend{Simple.jave, Faster.java, FasterQuad.java, simple.py}
	\end{axis}
\end{tikzpicture}

\bigskip\noindent

% graph
\begin{tikzpicture}
	\begin{axis}[
		title={Log table of average running time of 5 runs},
		xlabel={$N$},
    xmode = log,
    log ticks with fixed point,
    ymode = log,
		ylabel={Time (s)},
		xmin=30, xmax=3200,
		ymin=.01, ymax=160,
		xtick={30, 50, 100, 200, 400, 800, 1600, 3200},
		ytick={0.01, 0.05, 0.1, 0.5, 1, 2, 10, 50, 100, 160},
    legend style={at={(1.1,0)}, anchor=south west},
		ymajorgrids=true,
	]
  \addplot[ color=blue, mark=x,error bars/.cd,y dir=both,y explicit ] table [x index=0, y index=1] {output/tex_tables/Simple.table};
  \addplot[ color=red,  mark=*,error bars/.cd,y dir=both,y explicit ]	table [x index=0, y index=1] {output/tex_tables/Faster.table};
  \addplot[ color=green,mark=o,error bars/.cd,y dir=both,y explicit ] table [x index=0, y index=1] {output/tex_tables/FasterQuad.table};
  \addplot[ color=orange,mark=o,error bars/.cd,y dir=both,y explicit ] table [x index=0, y index=1] {output/tex_tables/Simple_python.table};
	\legend{Simple.jave, Faster.java, FasterQuad.java, simple.py}
	\end{axis}
\end{tikzpicture}

\end{document}
